\documentclass{practice}

\title{12}
\subtitle{Digital signatures \& elliptic curves}
\date{\DTMdate{2024-11-27}}

\begin{document}
\maketitle

\begin{task}{Correctness of DSA}
  Show that the DSA signature algorithm is correct.
  That is, than an honest verifier will always accept a correctly computed signature.
\end{task}

\begin{task}{DSA}
  Let the hash function $H$ be defined as $H(a) = (q - a) \bmod{q}$, where $q = 11$.

  Let $p = 23$ be a safe prime, and let $g = 2$ be a generator of $\GG$ in $\ZZ_{23}$ with $q$ elements.
  We consider DSA signatures in $\GG$.
  \begin{enumerate}
    \item Given the signing key $x = 6$, what is Alice's verification key $y$?
    \item Let $m = 3$ be Alice's message, and $k = 5$ be her encryption randomness.
    What is her signature on the message?
    \item Show how Bob would verify Alice's signature.
  \end{enumerate}
\end{task}

\begin{task}{Double trouble}
  Show that if the DSA signing randomness is re-used, an adversary can recover the signing key after collecting two message-signature pairs.
\end{task}

\begin{task}{Schnorr signature}
  Let the hash function $H$ be defined as $H(a,b) = ab \bmod{q}$, where $q = 11$.
  For the sake of simplicity in this task, we will consider that $H(a||b) \stackrel{\mathit{def.}}{=} H(a, b)$.

  We consider Schnorr signatures in $\GG$ for which we use the same group as for task 2.
  \begin{itemize}
    \item Given the signing key $x = 6$, what is Alice's verification key $y$?
    \item Let $m = 3$ be Alice's message, and $k = 5$ be her encryption randomness.
    What is her signature on the message?
    \item Show how Bob would verify Alice's signature.
  \end{itemize}
\end{task}

\newpage

\begin{task}{ECDSA}
  Let the hash function $H$ be defined as $H(a) = (n - a) \bmod{n}$, where $n = 7$.

  Consider the elliptic curve $E:$ $y^2 = x^3 + 2x + 7$ over the field $\FF_{11}$.
  Let $G = (6, 2)$ be a generator of $\GG$ of prime order $n$.
  We consider ECDSA signatures in $\GG$.
  \begin{enumerate}
    \item What is the order $n$ of the group generated by $G$? % 7
    \item Given the signing key $x = 4$, what is Alice's verification key $Y$?
    \item Let $m = 3$ be Alice's message, and $k = 2$ be her encryption randomness.
    What is her signature on the message?
    \item Show how Bob would verify Alice's signature.

    Note that if $u_1G + u_2Y = \pif$, the signature is invalid.
    Bob should also verify that
    \begin{enumerate}
      \item $Y \neq \pif$,
      \item $Y$ lies on the curve,
      \item $nY = \pif$.
    \end{enumerate}
  \end{enumerate}

  \begin{table}[h]
    \centering
    \begin{tabular}{c|ccccccc}
      \toprule
      $+$      & $\pif$   & $(6,2)$  & $(6,9)$  & $(7,1)$  & $(7,10)$ & $(10,2)$ & $(10,9)$\\\midrule
      $\pif$   & $\pif$   & $(6,2)$  & $(6,9)$  & $(7,1)$  & $(7,10)$ & $(10,2)$ & $(10,9)$\\
      $(6,2)$  & $(6,2)$  & $(10,9)$ & $\pif$   & $(10,2)$ & $(7,1)$  & $(6,9)$  & $(7,10)$\\
      $(6,9)$  & $(6,9)$  & $\pif$   & $(10,2)$ & $(7,10)$ & $(10,9)$ & $(7,1)$  & $(6,2)$\\
      $(7,1)$  & $(7,1)$  & $(10,2)$ & $(7,10)$ & $(6,2)$  & $\pif$   & $(10,9)$ & $(6,9)$\\
      $(7,10)$ & $(7,10)$ & $(7,1)$  & $(10,9)$ & $\pif$   & $(6,9)$  & $(6,2)$  & $(10,2)$\\
      $(10,2)$ & $(10,2)$ & $(6,9)$  & $(7,1)$  & $(10,9)$ & $(6,2)$  & $(7,10)$ & $\pif$\\
      $(10,9)$ & $(10,9)$ & $(7,10)$ & $(6,2)$  & $(6,9)$  & $(10,2)$ & $\pif$   & $(7,1)$\\\bottomrule
    \end{tabular}
    \caption{Point additions on $E$ over $\FF_{11}$.}
  \end{table}
\end{task}

\newpage

\begin{center}
  Concept refresher
\end{center}

\begin{tcolorbox}[title=Digital Signature Algorithm (DSA)]
  Parameter generation:
  \begin{itemize}
    \item Let $p = 2q + 1$ be a large safe prime, and let $\langle g\rangle = \GG \subsetneq \ZZ_p$ with $\lvert \GG\rvert = q$.
    \item Select $h \getsu \{2, \dots, p-2\}$ and set $g \gets h^{(p-1)/2}$.
    If $g = 1$, resample $h$ and try again.
    \item Randomly sample $x \getsu \{1, \dots, q-1\}$ and set $y \gets g^x$.
    \item The private signing key is $(p, q, g, x)$ and the public verification key is $(p, q, g, y)$.
  \end{itemize}

  Signing a message $m$:
  \begin{itemize}
    \item Sample $k \getsu \{1, \dots, q-1\}$ and set $r \gets \bigl(g^k \bmod{p}\bigr) \bmod{q}$.
    If $r = 0$, resample $k$ and try again.
    \item Compute $s \gets \bigl(k^{-1}(H(m)+xr)\bigr) \bmod{q}$.
    If $s = 0$, resample $k$ and try again.
    \item The signature on $m$ is $\sigma = (r, s)$.
  \end{itemize}

  Verifying a signature $\sigma = (r, s)$ on a message $m$:
  \begin{itemize}
    \item Verify that $r, s \in \{1, \dots, q-1\}$.
    \item Compute $w \gets s^{-1}\bmod{q}$.
    \item Compute $u_1 \gets \bigl(H(m) \cdot w\bigr)\bmod{q}$ and $u_2 \gets (r\cdot w)\bmod{q}$.
    \item Compute $v \gets \bigl(g^{u_1}y^{u_2} \bmod{p}\bigr)\bmod{q}$.
    \item The signature $\sigma$ on $m$ is valid only if $v = r$.
    
    \iffalse
    Indeed, notice that
    \[
      k \equiv H(m)s^{-1} + xrs^{-1}
        \equiv H(m)w + xrw \pmod{q}.
    \]
    Since $g$ has order $q$, we have
    \[
      g^k \equiv g^{H(m)w + xrw} \equiv g^{H(m)w} \cdot y^{rw} \equiv g^{u_1} \cdot y^{u_2} \pmod{q}.
    \]
    Finally,
    \[
      r = \bigl(g^k \bmod{p}\bigr)\bmod{q} = (g^{u_1} \cdot y^{u_2} \bmod{p}) \bmod{q} = v.
    \]
    \fi
  \end{itemize}

  Note that the hash function should be picked such that $H(m) \in \ZZ_q$.
  Furthermore, $H$ must be a cryptographic hash function.
  \tcblower
  FIPS 186-5 has obsoleted the use of DSA for creating new signatures, but ECDSA is still allowed.
\end{tcolorbox}

\begin{tcolorbox}[title=Elliptic curves]
  Elliptic curves are a family of curves which are given by the equation
  \[
    y^2 = x^3 + ax + b.
  \]
  This form is called the \emph{Weierstrass form}.
  Elliptic curves can also be represented by different, more specific forms, e.g. Montgomery form.

  In cryptography, we consider elliptic curves defined over a finite field $\FF_p$ for some prime $p$.
  The set of points on a curve $E$ denoted by $E(\FF_p)$ has cardinality $n$.
  In cryptography, we often want $n$ to be prime.

  \tcblower

  Some generic formulae:
  \begin{itemize}
    \item Point addition
    
    Let $P_1 = (x_1, y_1), P_2 = (x_2, y_2)$ be two \emph{distinct} points on $E$.

    We compute $P_3 = P_1 + P_2$ with
    \begin{itemize}
      \item $\lambda = (y_2 - y_1)(x_2 - x_1)^{-1}$
      \item $x_3 = \lambda^2 - x_1 - x_2$
      \item $y_3 = \lambda(x_3 - x_1) - y_1$
    \end{itemize}

    \item Point doubling
    
    Compute $Q = P + P = 2P$ as above, except with 
    $\lambda = \frac{3x^2 + a}{2y}$.

    \item Point negation
    
    To find $-P$ such that $P + (-P) = \pif$, negate the $y$ coordinate of $P$.
  \end{itemize}
\end{tcolorbox}

\newpage

\begin{tcolorbox}[title=Schnorr group]
  A Schnorr group is a prime-order subgroup of $\ZZ_p^*$.
  That is, a group $\GG$ of prime-order $q$ such that $p=rq + 1$ is prime.
  When $r = 2$, $p$ is called a \emph{safe prime} and $q$ a \emph{Sophie-Germain} prime.
\end{tcolorbox}

\begin{tcolorbox}[title=Schnorr signature]
  Let $\GG$ be a Schnorr group of prime-order $q$ generated by $g$ where the DLP is hard.

  Key generation:
  \begin{itemize}
    \item Randomly sample $x \getsu \{1, \dots, q-1\}$ and set $y \gets g^{x}$.
    \item The private signing key is $(p, q, g, x)$ and the public verification key is $(p, q, g, y)$.
  \end{itemize}

  Signing a message $m$:
  \begin{itemize}
    \item Sample $k \getsu\{1, \dots, q-1\}$ and set $r \gets g^k$.
    \item Compute $e \gets H(m||r) \bmod{q}$
    \item Compute $s \gets (k - ex) \bmod{q}$.
    \item The signature on $m$ is $\sigma = (s, e)$.
  \end{itemize}

  Verifying a signature $\sigma = (r, s)$ on a message $m$:
  \begin{itemize}
    \item The signature $\sigma$ on $m$ is valid only if $e = H(m||g^s \cdot y^e)$
    
    Indeed,
    \[
      g^s \cdot y^e = g^{k - ex} \cdot g^{ex} = g^k = r
    \]
    and so $H(m||g^s\cdot y^e) = H(m||r) = e$.
  \end{itemize}
  \tcblower
  Similarly to DSA/ECDSA, reusing the nonce $k$ leads to private key retrieval.
\end{tcolorbox}

\end{document}
