\documentclass{practice}

\usepackage{pgfplots}
\pgfplotsset{width=20cm,compat=1.18}
\usepgfplotslibrary{external,fillbetween}

\newcommand*{\MAC}{\mathsf{MAC}}

\usepackage{subcaption}

\usepackage[dvipsnames,hyperref]{xcolor}

\begin{document}
\begin{center}
  \textbf{Practice 13, solutions}
\end{center}

\begin{task}{Simple identification}
  The protocols for public key encryption and digital signatures are given in the slides.

  \begin{figure*}[h!]
    \centering
    \subfloat[Symmetric encryption]{
      \begin{tabularx}{6.5cm}{@{}lXl@{}}
        \textbf{Prover} & & \textbf{Verifier}\\
        \midrule
        & & $c \getsu \{0,1\}^k$\\
        &\begin{tikzpicture}[trim left=0cm,trim right=\linewidth] % https://tex.stackexchange.com/a/58359
            \node (A) at (0, 0) {};
            \node (B) at (\linewidth,0) {};
            \draw[<-] (A) -- (B) node[above,midway] {$c$};
          \end{tikzpicture}&\\
        $r \gets \Enc_{\sk}(c)$\\
        &\begin{tikzpicture}[trim left=0cm,trim right=\linewidth]
            \node (A) at (0, 0) {};
            \node (B) at (\linewidth-2\pgflinewidth, 0) {};
            \draw[->] (A) -- (B) node[above,midway] {$r$};
          \end{tikzpicture}&\\
        & & $c \iseq \Dec_{\sk}(r)$
      \end{tabularx}
    }
    \qquad
    \subfloat[MAC]{
      \begin{tabularx}{6.5cm}{@{}lXl@{}}
        \textbf{Prover} & & \textbf{Verifier}\\
        \midrule
        & & $c \getsu \{0,1\}^k$\\
        &\begin{tikzpicture}[trim left=0cm,trim right=\linewidth] % https://tex.stackexchange.com/a/58359
            \node (A) at (0, 0) {};
            \node (B) at (\linewidth,0) {};
            \draw[<-] (A) -- (B) node[above,midway] {$c$};
          \end{tikzpicture}&\\
        $r \gets \MAC_{\sk}(c)$\\
        &\begin{tikzpicture}[trim left=0cm,trim right=\linewidth]
            \node (A) at (0, 0) {};
            \node (B) at (\linewidth-2\pgflinewidth, 0) {};
            \draw[->] (A) -- (B) node[above,midway] {$r$};
          \end{tikzpicture}&\\
        & & $r \iseq \MAC_{\sk}(c)$
      \end{tabularx}
    }
    \end{figure*}
\end{task}

\begin{task}{Evaluating properties}
  \begin{enumerate}
    \item Completeness
    
    The protocol is complete if for a true statement, an honest verifier accepts the proof of an honest prover.
    That is, a proof for a true statement should always verify correctly.
    \item Soundness
    
    The protocol is sound if an honest verifier does not accept the proof for a false statement made by a dishonest prover, except with some negligible probability.
    That is, it should not be possible to prove wrong things.
    \item Zero-knowledge
    
    The protocol has zero knowledge if a (potentially dishonest) verifier cannot learn anything from the proof of a true statement, other than that the statement is true.
  \end{enumerate}
\end{task}

\begin{task}{Schnorr's identification protocol}
  \begin{enumerate}
    \item The protocol is complete.
    
    \begin{proof}
      \[
        g^r = g^{u + cx} = g^u \cdot g^{cx} = a\cdot \bigl(g^{x}\bigr)^c = a\cdot h^c,
      \]
      where $g, r, a, h, c$ are all known to the verifier.
    \end{proof}

    \item The protocol is special sound.
    
    \begin{proof}
      Let $(a, c, r)$, $(a, c', r')$ be two accepting transcripts such that $c \neq c'$.
      This implies that $r \neq r'$ since $x, u$ are the same when computing $r$ and $r'$.
      Then
      \[
        r - r' = u + cx - u - c'x = x(c - c')
      \]
      and so $x = (r - r')/(c - c')$.

      It follows that the prover actually knows the discrete logarithm $x = \log_g(h)$ since from
      \[
        g^r = ah^c,\qquad g^{r'} = ah^{c'}
      \]
      and so
      \[
        h = g^{(r-r')/(c - c')}.
      \]
      The discrete logarithm can be computed since $r, r', c, c'$ are known and the inverse of $c - c'$ is efficiently computable as the modulus is known, e.g. with the EEA.

      In more detail, since $a = g^r/h^c = g^{r'}/h^{c'}$ we have that
      \begin{alignat*}{2}
        && g^{r'} \cdot h^{-c'} &= g^r \cdot h^{-c} \\
        \Leftrightarrow\quad&& h^c \cdot h^{-c'} &= g^r \cdot g^{-r'}\\
        \Leftrightarrow\quad&& h^{c - c'} &= g^{r - r'} \\
        \Leftrightarrow\quad&& h^{c - c'} \cdot h^{1/(c - c')} &= g^{r - r'} \cdot h^{1/(c - c')} \\
        \Leftrightarrow\quad&& h &= g^{(r-r')/(c - c')}.
      \end{alignat*}
    \end{proof}

    Intuitively, after sending the announcement $a$, a prover that does not know $x$ can answer at most one challenge correctly.
    That is, if the prover guesses the challenge correctly, it can craft a commitment and response such that the transcript is accepting\footnotemark{}.
    This happens with probability $1/q$.
    The protocol is thus sound with soundness error $1/q$.
    \footnotetext{See the simulation principle in the proof of the zero-knowledge property.}

    \item The protocol is special honest-verifier zero-knowledge.
    
    \begin{proof}
      In the honest protocol execution, $u$ is sampled uniformly at random and so $a \gets g^u$ is randomly distributed.
      $c$ is also sampled uniformly at random, and thus $r \gets u + cx \pmod{q}$ is randomly distributed, since $u, c$ are randomly distributed.

      When not bound by the sequential order of the protocol steps, an accepting transcript can be generated with:
      \begin{enumerate}
        \item $r \getsu \ZZ_q$
        \item $c \getsu \ZZ_q$
        \item $a \gets g^r\cdot h^{-c}$
      \end{enumerate}

      Indeed, after having picked a response and the challenge, the simulator must pick the commitment such that the verification equation would hold:
      \[
        g^r = ah^c
        \qquad\iff\qquad
        g^r h^{-c} = a.
      \]
      Since $r, c$ are distributed uniformly at random, so is $a$.
      Each of the three steps is efficient.

      It follows that all components in the transcripts of both real and simulated conversations are uniformly distributed, and thus the distributions are identical.
    \end{proof}

    If the verifier did not pick the challenge at random, then the simulator would not be efficient.
    Indeed, the probability that the simulator's randomly sampled challenge would correspond to the specific challenge of a dishonest verifier would be $1/q$, as the simulator samples challenges at random.
    Note that since we do not know the challenge generation procedure of the dishonest verifier, the simulator cannot simulate it/pick challenges following the same algorithm.

    This means that $q$ tries are needed on average to find an accepting transcript which would match the probability distribution of a transcript of a real conversation between the honest prover and dishonest verifier.
    The running time of the simulator is thus $\Oh(q)$ which is exponentially large with respect to the security parameter/bit-length of group size.
    I.e. if $q$ is $\lambda$ bits long, then the simulator running time is $\Oh\bigl(2^\lambda\bigr)$.
  \end{enumerate}
\end{task}

\begin{task}{Fiat-Shamir identification protocol}
  \begin{tcolorbox}[title=Warning!]
    The exercise I have given you is misleading (has to do with QRs and NQRs) as I forgot that we are working modulo a composite, and not modulo a prime.
    Modulo a composite, the product of two NQRs might not be a QR.
    Likewise, the product of an NQR and a QR might be a QR.
    \tcblower
    See slide 15 for the correct soundness proof.
    The soundness proof below is presented with the same misleading assumptions.
  \end{tcolorbox}

  \begin{enumerate}
    \item The protocol is complete.
    \[
      r^2 \equiv \bigl(x^cu\bigr)^2 \equiv x^{2c}u^2 \equiv \bigl(x^2\bigr)^ca \equiv y^ca \pmod{n},
    \]
    where $r, y, c, a$ are all known to the verifier

    \item The protocol is $1/2$-sound.
    
    \begin{itemize}
      \item When $c = 0$ we have for an honest execution that $r \gets u$ and since $r^2 = u^2 = a$, the verifier gets the guarantee that the announcement $a$ made by the prover is indeed a quadratic residue.
      Knowing that the prover indeed announces some quadratic residue $a$ is important information to the verifier.
      However, when $c = 0$, the verification is independent of the witness $x$.
      \item When $c = 1$ we have for an honest execution that $r \gets xu$ and since $r^2 = x^2u^2 = ya$ the verifier gets the guarantee that $ya$ is a quadratic residue.
      
      The verifier can be certain that $r^2$ is a quadratic residue since it computes it from $r$ and so the property follows by definition.
      In turn, this implies that either $y$ and $a$ are quadratic residues or that neither $y$ nor $a$ are quadratic residues\footnotemark{}.
      \footnotetext{The product of two QRs or of two NQRs is a QR. Only the product of a NQR and a QR is a NQR.}

      The fact that $y$ is a QR is what the proof is supposed to prove, and $y$ must be a QR if $a$ is a QR.
      Thus, the verifier needs a way to verify that $a$ is a quadratic residue, which is done exactly by the case where $c = 0$.

      If the verifier did not check whether $a$ is indeed a quadratic residue, the prover could set $a$ to be a NQR when $y$ is a NQR.
      The verification equation would then hold, and the verifier would think that $y$ is a QR when this is not the case, thus making the protocol unsound.

      Since the prover must announce $a$ before the verifier decides whether it wants to verify whether $y$ or $a$ is a QR, it cannot cheat without the risk of getting caught.
      The prover could try to guess for each protocol run whether the challenger picks $c = 0$ or $c = 1$ but the probability of guessing this correctly after $k$ protocol runs is $2^{-k}$ which can be made as small as needed.

      For example, let $y$ be an NQR.
      In an attempt to convince the verifier that $y$ is a QR, the prover selects some NQR $a$ and hopes that the verifier selects $c = 1$.
      However, if the verifier now selects $c = 0$, then $r^2 \not\equiv a$ and the verifier sees that the prover does not announce $a$ correctly and refuses to deal with the prover further.

      In summary, the verifier is convinced with probability $1/2$ that $a$ is a QR.
      By successfully repeating the protocol $k$ times, the probability increases.
      If the verifier is convinced that $a$ is a QR, then the check $r^2 \equiv ya \pmod{n}$ convinces the verifier that $y$ is a QR.
      This check happens with probability $1/2$ and so the protocol is $1/2$-sound.

      \textit{You can make this proof more formal by using a similar approach to the one in the previous exercise.
      I.e., show what happens for two transcripts with the same $a$ value, but $c = 0$ and $c' = 1$.
      Alternatively, see the proof in slide 15.}
    \end{itemize}

    \item The protocol is honest-verifier zero-knowledge.
    
    In the real conversation, all transcript components are uniformly distributed.
    When not bound by the sequential order of the protocol steps, an accepting transcript can be generated with:
    \begin{enumerate}
      \item $r \getsu \ZZ_n$
      \item $c \getsu \{0, 1\}$
      \item $a \gets r^2\cdot y^{-c}$
    \end{enumerate}
    Since the modulus $n$ is known, the inverse of $y$ can be computed efficiently, e.g. with the EEA, without needing to know the factorisation of $n$.
    Note that $y$ has an inverse as long as $p \neq y \neq q$.
    If $y = p$ or $y = q$, there is no zero-knowledge proof to be made, as the verifier is already certain that $y$ is not a QR.
    Since $r,c$ are distributed at random, so is $a$.

    When $c = 0$, $y^0a \equiv a \equiv r^2 y^{0} \equiv r^2 \pmod{n}$ and the verification succeeds.
    When $c = 1$, $ay^1 \equiv r^2y^{-1}y^1 \equiv r^2 \pmod{n}$ and the verification succeeds.

    It follows that all components in the transcripts of both real and simulated conversations are uniformly distributed, and thus the distributions are identical.

    If the verifier does not pick the challenge at random, then the simulator can still efficiently simualte the proof.
    The probability that the simulator's randomly sampled challenge corresponds to the specific challenge of a dishonest verifier is $1/2$, as the simulator samples challenges at random.

    This means that on average, it takes $2$ tries to match the probability distribution of the transcript of a real conversation between the honest prover and dishonest verifier.
    The running time of the simulator thus remains efficient.
  \end{enumerate}
\end{task}

\begin{task}{Sounds like cheating}
  Let the prover be malicious and attempt to convince the verifier that she knows the square root of $y$ when this is not the case.
  The prover can then use the standard simulation technique to compute $r$ and $a$, and then adjust the values if necessary:
  \begin{enumerate}
    \item The prover samples $r \getsu \ZZ_n$
    \item The prover samples $c \getsu \{0,1\}$
    \item The prover sets $a \gets r^2y^{-c}$
    \item If $a \bmod 2 \neq c$, the prover sets $c \gets 1 - c$ (changes $0$ to $1$ or vice versa) and recomputes $a$.
    The following cases illustrate why $a \bmod 2 = c$ is required and why the prover can cheat:
    \begin{enumerate}
      \item Let $a$ be even, i.e. $a \bmod 2 = 0$.
      \begin{itemize}
        \item If $c = 0$ then the verification $ay^0 \equiv a \equiv r^2y^{0} \equiv r^2 \pmod{n}$ succeeds.
        \item If $c = 1$ then the verification $ay^1 \equiv r^2y^{0}y^1 \equiv r^2y \not\equiv r^2 \pmod{n}$ fails, which is why $c = 0$ is needed.
      \end{itemize}
      \item Let $a$ be odd, i.e. $a \bmod 2 = 1$.
      \begin{itemize}
        \item If $c = 0$ then the verification $ay^0 \equiv a \equiv r^2y^{-1} \not\equiv r^2 \pmod{n}$ fails, which is why $c = 1$ is needed.
        \item If $c = 1$ then the verification $ay^1 \equiv r^2y^{-1}y^1 \equiv r^2 \pmod{n}$ succeeds.
      \end{itemize}
    \end{enumerate}

    \item If $a \bmod 2 \neq c$ still, the prover starts over from step 1.
  \end{enumerate}

  Since the prover can anticipate the challenge value of the prover, it can simualte accepting transcripts, and use the $a, r$ values in the real protocol execution to always convince the verifier. 
\end{task}

\end{document}