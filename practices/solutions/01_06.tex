\documentclass{practice}

\usepackage{cases}

\begin{document}

\begin{center}
  \textbf{Practice 1, task 6 solution}
\end{center}

\begin{table}[h!]
  \centering
  \begin{tabular}{@{}cccccccccccccc@{}}
    A  & B  & C  & D  & E  & F  & G  & H  & I  & J  & K  & L  & M & \dots\\\midrule
    0  & 1  & 2  & 3  & 4  & 5  & 6  & 7  & 8  & 9  & 10 & 11 & 12 & \dots\\\\
    \dots & N  & O  & P  & Q  & R  & S  & T  & U  & V  & W  & X  & Y  & Z\\\midrule
    \dots & 13 & 14 & 15 & 16 & 17 & 18 & 19 & 20 & 21 & 22 & 23 & 24 & 25
  \end{tabular}
  \caption{Letter positions of the English alphabet (0-indexed).}
  \label{table:alphabet}
\end{table}

To conduct a KPA against the affine cipher, we must think about the plaintext and ciphertext in mathematical terms.
As such, let us first convert the plaintext and ciphertext into their numeric forms by replacing each letter with the letter's index in the alphabet.
The index-letter mapping for the English alphabet is given in \autoref{table:alphabet}.

\begin{figure}[h!]
  \centering
  \begin{tabular}{@{}C{1em}C{1em}C{1em}C{1em}C{1em}C{1em}@{}}
    c & i & p & h & e & r\\\midrule
    2 & 8 & 15 & 7 & 4 & 17
  \end{tabular}
  \hspace*{1cm}
  \begin{tabular}{@{}C{1em}C{1em}C{1em}C{1em}C{1em}C{1em}@{}}
    s & w & f & r & c & p\\\midrule
    18 & 22 & 5 & 17 & 2 & 15
  \end{tabular}
\end{figure}

Each ciphertext letter $y_i$ is obtained by applying the encryption function to the corresponding plaintext letter $x_i$, where the subscript $i$ indicates the position of the letter in the string.
For example, in the known plaintext-ciphertext pair:
\begin{itemize}
  \item $x_1 = 2$, $y_1 = 18$ represent the first letter of the plaintext and of the ciphertext,
  \item $x_6 = 17$, $y_6 = 15$ represent the last letter of the plaintext and of the ciphertext.
\end{itemize}

\begin{tcolorbox}
  Notice that I start indexing from 1 instead of 0.
  While I could have started indexing from 0, I find indexing from 1 to be more natural when talking about positions.
  The reason why we index the alphabet's letters from 0 is because we are working mod 26, and so starting from 0 is important for the calculations!
\end{tcolorbox}

By rewriting the encryption formula with the index notation, we get
\[
  y_i = \Enc_{(a,b)}(x_i) = (ax_i + b) \bmod{26}.
\]
For some concrete values, e.g. $x_1 = 2, y_1 = 18$, this means
\[
  18 = \Enc_{(a,b)}(2) = (2a + b) \bmod{26},
\]
where the tuple $(a, b)$ is the key which is unknown to us.
We say that $a,b$ are two unknowns.

You might recall that a system of $n$ linear equations\footnotemark{} with $n$ unknowns generally has a single solution if the equations are linearly independent from one another.
\footnotetext{\url{https://en.wikipedia.org/wiki/System_of_linear_equations}}%
Thus, if we can get two suitable plaintext-ciphertext letter combinations, we should be able to recover the two unknowns that make up the key.
Since the known pair is made of 6 letters, we should have enough material to work with.

\iffalse
You might notice that the affine function used for encryption can be graphed as a line (if we disregard the modulus).
You might then recall that two non-parallel lines will always intersect in exactly one point.
This hopefully provides some intuition to the fact that a system of two linear equations with two unknowns has a unique solution, provided that the equations are not multiples of one another.
\fi

Let us pick the pairs $(x_1, y_1) = (2, 18)$, $(x_6, y_6) = (17, 15)$ to try and solve the system.
We have
\begin{numcases}{}
  2a + b \equiv 18 \pmod{26}\label{eq:c11}\\
  17a + b \equiv 15 \pmod{26}\label{eq:c62}
\end{numcases}
If the \enquote{$\equiv$} notation seems strange to you, see the \nameref{sec:addendum}.

We cannot solve for two variables at once, so let us subtract \eqref{eq:c11} from \eqref{eq:c62} to cancel out the $b$:
\begin{alignat}{3}
  &&\quad
  (17a + b) - (2a + b) &\equiv 15 - 18 &&\pmod{26}\\
  \Leftrightarrow&&
  15a &\equiv -3 &&\pmod{26}\\
  \Leftrightarrow&&
  15a &\equiv 23 &&\pmod{26}\label{eq:invert-fifteen}
\end{alignat}
Substitution could work as well, but subtraction is less confusing here.

Since ordinary division is not defined in modular arithmetic, we must find the \emph{modular multiplicative inverse} (henceforth simply \enquote{inverse}) of $15$ modulo $26$.
For this we can use the \emph{extended Euclidean algorithm} (EEA).

First, we use the Euclidean algorithm (EA) to compute the greatest common divisor (GCD) of $15$ and $26$ and check whether $\gcd(15, 26)=1$.
If not, $15$ does not have an inverse modulo $26$.
If yes, we \enquote{backtrack} the algorithm by substituting the remainders to recover the \emph{Bézout coefficients} and thereby determine the inverse.
The Bézout coefficients $s, t$ are integers such that $\gcd(a,b) = sa + tb$.

Typically the EEA refers to the more efficient version of the algorithm\footnotemark{} which computes the coefficients and the GCD in parallel.
You can study it to speed up your calculations, but we will not cover it here.
\footnotetext{\url{https://en.wikipedia.org/wiki/Extended_Euclidean_algorithm}}%

\begin{tcolorbox}[title=Modular multiplicative inverse]
  The modular multiplicative inverse $x$ of an integer $a$ with respect to the modulus $m$ is an integer $x$ which satisfies
  \[
    ax \equiv 1 \pmod{m}.
  \]
  We often write the modular multiplicative inverse of $a$ as $a^{-1}$.
  Note that $-a$ is the \emph{additive inverse} of $a$.
  \tcblower
  For linguistic simplicity, I will use \enquote{inverse} when referring to the multiplicative inverse $a^{-1}$ (we use it more), and \enquote{additive inverse} when talking about $-a$.
\end{tcolorbox}

Computing the GCD with the Euclidean algorithm gives us:
\begin{alignat*}{3}
  26 &= 1 &&\cdot {\color{purple}15} &&+ {\color{blue}11}\\
  {\color{purple}15} &= 1 &&\cdot {\color{blue}11} &&+ {\color{teal}4}\\
  {\color{blue}11} &= 2 &&\cdot {\color{teal}4}  &&+ {\color{violet}3}\\
  {\color{teal}4}  &= 1 &&\cdot {\color{violet}3}  &&+ {\color{orange}1}\\
  {\color{violet}3}  &= 3 &&\cdot {\color{orange}1}  &&+ 0
\end{alignat*}
and the GCD is the last nonzero remainder, i.e. $\color{orange}1$.
We have $\gcd(15, 26)=1$ and so $15$ has an inverse which we will find by reversing the process.

This requires us to move upwards from the line that gave us the GCD, and isolate the terms to the right of the $+$ signs.
Rearranging the equations gives us:
\begin{alignat*}{3}
  {\color{orange}1} &= 4  &&- 1 &&\cdot 3\\
  {\color{violet}3} &= 11 &&- 2 &&\cdot 4\\
  {\color{teal}4}   &= 15 &&- 1 &&\cdot 11\\
  {\color{blue}11}  &= 26 &&- 1 &&\cdot 15
\end{alignat*}
Then, starting with the lowest number, we substitute all numbers that are not $15, 26$ or their coefficients.
Recall that the goal is to get the GCD in terms of coefficients of $15$ and $26$, i.e. to recover the Bézout coefficients.
We have
\begin{align}
  \gcd(15, 26) &= {\color{orange}1}
  = {\color{orange}4 - 1 \cdot 3}\\
  &= 4 - {\color{violet}3}
  = 4 - {\color{violet}(11 - 2 \cdot 4)}\label{eq:subthree}\\
  &= -11 + 3 \cdot {\color{teal}4}
  = -11 + 3 \cdot {\color{teal}(15 - 11)}\label{eq:plusthree}\\
  &= 3 \cdot 15 - 4 \cdot {\color{blue}11}
  = 3 \cdot 15 - 4 \cdot {\color{blue}(26 - 15)}\label{eq:minusfour}\\
  &= -4 \cdot 26 + 7 \cdot 15
\end{align}
and from
\[
  \gcd(15, 26) = 7 \cdot 15 - 4 \cdot 26
\]
we see that $7$ is the inverse of $15$ modulo $26$.

\begin{tcolorbox}[title=Tip]
  When reversing the EA, be careful to substitute only one remainder at a time and keep track of the \enquote{consumed} substitutions.
  For example, the $3$ in \eqref{eq:plusthree} or $-4$ in \eqref{eq:minusfour} are coefficients and you should not substitute them!
  $3$ was already substituted in \eqref{eq:subthree}, and $4$ in \eqref{eq:plusthree}.
  On paper, I like to write down which substitution I have already consumed.
\end{tcolorbox}

Now that we know the modular multiplicative inverse of $15$ which we write as $15^{-1}$, we can continue solving \eqref{eq:invert-fifteen}.
We have:
\begin{alignat*}{3}
  &&\quad
  15a \cdot 15^{-1} &\equiv 23 \cdot 15^{-1} &&\pmod{26}\\
  \Leftrightarrow&&
  a &\equiv 23 \cdot 7 &&\pmod{26}\\
  \Leftrightarrow&&
  a &\equiv 161 &&\pmod{26}\\
  \Leftrightarrow&&
  a &\equiv 5 &&\pmod{26}
\end{alignat*}

Now that we know $a$, we can substitute it in \eqref{eq:c11}, and so
\begin{alignat*}{3}
  &&\quad
  2 \cdot 5 + b &\equiv 18 &&\pmod{26}\\
  \Leftrightarrow&&
  b &\equiv 18 - 10 &&\pmod{26}\\
  \Leftrightarrow&&
  b &\equiv 8 &&\pmod{26}
\end{alignat*}

We have now recovered the key $a = 5, b = 8$, and can decrypt each ciphertext letter $y_i$ with
\[
  x_i = \Dec_{(5,8)}(y_i) = 5^{-1}(y_i - 8) \bmod{26}
  = 21(y_i - 8) \bmod{26}.
\]

Indeed, we have that
\[
  26 = 5 \cdot 5 + 1
\]
and so
\[
  \gcd(5, 26) = 1 = 26 - 5 \cdot 5
\]
and the inverse of $5$ is $-5 \bmod{26} = 21$.

We can then decrypt the challenge ciphertext:
\begin{alignat*}{3}
  x_1 &= 21(13 &&- 8) \bmod 26 = 105  &&\bmod 26 = 1\\
  x_2 &= 21(15 &&- 8) \bmod 26 = 147  &&\bmod 26 = 17\\
  x_3 &= 21(0  &&- 8)  \bmod 26 = -168 &&\bmod 26 = 14\\
  x_4 &= 21(6  &&- 8)  \bmod 26 = -42  &&\bmod 26 = 10\\
  x_5 &= 21(2  &&- 8)  \bmod 26 = -126 &&\bmod 26 = 4\\
  x_6 &= 21(21 &&- 8) \bmod 26 = 273  &&\bmod 26 = 13
\end{alignat*}
and we recover the word \enquote{broken}.

\begin{figure}[h!]
  \centering
  \begin{tabular}{@{}C{1em}C{1em}C{1em}C{1em}C{1em}C{1em}@{}}
    n & p & a & g & c & v\\\midrule
    13 & 15 & 0 & 6 & 2 & 21
  \end{tabular}
  \hspace*{1cm}
  \begin{tabular}{@{}C{1em}C{1em}C{1em}C{1em}C{1em}C{1em}@{}}
    b & r & o & k & e & n\\\midrule
    1 & 17 & 14 & 10 & 4 & 13
  \end{tabular}
\end{figure}

Note that there are easier/quicker pairs with which to solve the exercise, e.g. (c, s) and (e, c), but the chosen approach illustrates the general method.

\section*{Addendum}\label{sec:addendum}

Let $\ZZ$ be the set of integers, i.e. $\ZZ = \{\dots, -2, -1, 0, 1, 2, \dots\}$, and let $a, b, m \in \ZZ$.

The notation
\[
  a \equiv b \pmod{m}
\]
denotes that $a,b$ are \emph{congruent} modulo $m$.
The symbol $\equiv$ is the \emph{equivalence} symbol.
The congruence relation can be written with a normal equality as
\[
  a - b = km
\]
for some $k\in\ZZ$.
From this we conclude that $a,b$ are congruent modulo $m$ when $m$ divides $a - b$, which is written as $m\mid(a-b)$.
Indeed,
\[
  a - b = km \qquad\iff\qquad
  (a-b)/m = k
\]
and since we fixed $k$ to be an integer, $m$ has to perfectly divide $a-b$.
What the precise value of $k$ is is not important in a congruence relation.

The notation
\[
  a \bmod m = b
\]
is not well liked by mathematicians but is frequent in computer science since it illustrates how we call the modulo operations in code, e.g. \texttt{int b = a \% b} with \texttt{\%} being the modulo operator in many languages.
Note that
\[
  a \equiv b \pmod{m} \qquad\iff\qquad
  a \bmod{m} = b
\]
assuming that $b \le a$.

\end{document}
