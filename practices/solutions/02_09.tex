\documentclass{practice}

\begin{document}
\begin{center}
  \textbf{Practice 2, task 9 solution}
\end{center}

Compute the unicity distance of the following ciphers:
\begin{itemize}
  \item Caesar's cipher
  \item Affine cipher
  \item Vigenere cipher (in terms of key-length $n$)
\end{itemize}

\vspace*{1em}

\textbf{Unicity distance}

Recall that the unicity distance describes the amount of ciphertext needed to be confident that we have recovered the correct plaintext.
Indeed, for a short message, it is possible that multiple keys give a meaningful plaintext, and we are faced with the same problem as with the one-time pad: which message is correct?
Thus, if the ciphertext is longer than the unicity distance, there should be just one decryption which is meaningful.

\textbf{Note.}
The unicity distance is only a theoretical minimum, and in general we might need much more characters to reliably break a cipher!

More formally, the unicity distance helps us reduce the number of \emph{spurious keys}, i.e. keys that do not give us the correct plaintext, assuming that the plaintext has redundancy.
The unicity distance $N_0$ is computed with the formula
\[
  N_0 = \frac{H(K)}{D},\qquad
  H(K) = \log_2{\lvert K \rvert}
\]
where $H(K)$ is the entropy of the key space $K$, and $D$ is the plaintext redundancy in bits per character.
Recall that $\lvert K\rvert$ denote the cardinality of the key space, i.e. the number of possible keys.

Each character in English can convey $\log_2(26) \approx 4.7$ bits of information.
However, meaningful English texts contain only about $1.5$ bits of information per character.
Thus, the English plaintext redundancy is $D_\mathit{English} = 4.7 - 1.5 = 3.2$.

\textbf{Affine cipher}

To compute the unicity distance we need to find the number of possible keys for the (English) Affine cipher.

\begin{itemize}
  \item For $b$ the answer is straightforward.
  We are working modulo $26$, so all values $0, \dots, 25$ work for $b$.
  \item For $a$ the answer is slightly more complex.
  Recall that for decryption to work, $a$ must be picked such that $\gcd(a, 26) = 1$.
  We can either find the number of such integers by hand since the values are small, or we can use Euler's totient function.

  Indeed, Euler's \emph{totient} function $\varphi(n)$ counts the positive integers up to a given integer $n$ that are relatively prime (coprime) to $n$.
  Sometimes it is also called Euler's \emph{phi} function since it is written using the Greek letter phi.
  One way to compute $\varphi(n)$ is with the \emph{prime factorisation}\footnotemark{} of $n$.
  \footnotetext{Any number can be expressed as the product of distinct primes, e.g. $360 = 2^3 \cdot 3^2 \cdot 5$.
  It is not an easy problem to find the prime factorisation.
  In fact, the security of many cryptographic algorithms (e.g. RSA) relies on the difficulty of prime factorisation for large numbers.}%
  Euler's totient function is then computed as
  \[
    \varphi(n) = p_1^{k_1 - 1}(p_1 - 1) p_2^{k_2 - 1}(p_2 - 1) \dots p_r^{k_r - 1}(p_r - 1)
  \]
  where $n = p_1^{k_1}p_2^{k_2}\dots p_r^{k_r}$ is the prime factorisation of $n$.

  We then have
  \[
    \varphi(26) = \varphi(2 \cdot 13) = \varphi(2) \cdot \varphi(13) = (2-1) \cdot (13 - 1) = 12
  \]
  and so there are $12$ possible values which $a$ can take.
  If you were to find the possible values by hand, $a$ can be anything that is not divisible by $2$ or $13$.
  Therefore,
  \[
    a\in\{1, 3, 5, 7, 9, 11, 15, 17, 19, 21, 23, 25\}
  \]
  which is indeed $12$ possibilities.
\end{itemize}

The total number of possible keys is thus $12 \cdot 26 = 312$ since there are $12$ ways to pick $a$, $26$ ways to pick $b$, and the pics are independent.
Therefore $N_0 = \frac{\log_2(312)}{3.2} \approx 2.59$.
Since we are talking about the length of a ciphertext, we have to round up since we cannot have partial character lengths.
The unicity distance of the Affine cipher is thus $3$.

\vspace*{1em}

\textbf{Vigenere cipher}

For the Vigenere cipher, the size of the key space varies and so the unicity distance is different per key length.
When using Vigenere with the English alphabet, the size of the key space is thus $26^n$, where $n$ denotes the length of the key.
Indeed, for each letter of the key we have $26$ options, and we can reuse the same letter at multiple positions.

We then have
\[
  N_0 = \frac{\log_2\bigl(26^n\bigr)}{3.2} = \frac{n\log_2(26)}{3.2} \approx 1.47n.
\]

\end{document}