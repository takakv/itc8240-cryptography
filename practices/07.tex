\documentclass{practice}

\usepackage{amsthm}
\usepackage{crysymb}

\usepackage{tikz}
\usetikzlibrary{calc,decorations.pathmorphing,shapes,positioning,patterns,shadows.blur}

\usepackage{algorithm}
\usepackage{algpseudocode}

\title{7}
\subtitle{Limited adversaries}
\date{\DTMdate{2024-10-23}}

\begin{document}
\maketitle

If not specified otherwise, we treat $\GG$ as a multiplicative group of prime order, i.e. the group contains a prime number of elements, is cyclic, and its operation is multiplication.

\begin{task}{CDH from DL}
  Show that if we have access to an adversary which can solve the DL problem, we can use it to solve the CDH problem.
\end{task}

\begin{task}{DDH from DL}
  Show that if we have access to an adversary which can solve the CDH problem, we can use it to solve the DDH problem.
\end{task}

\begin{task}{Reduce and rejoice}
  % https://crypto.stackexchange.com/a/82042
  % https://crypto.stackexchange.com/a/27161
  Consider the following problem which we will call the Square Diffie-Hellman (SDH) problem.
  Let $\langle g\rangle = \GG$ be a group of prime-order $q$.
  Given $(g, g^a)$, find $g^{a^2}$.

  Show that
  \begin{enumerate}
    \item If we have access to $\AD$ which can solve the CDH problem, we can use it to solve the SDH problem.
    \item If we have access to $\AD$ which can solve the SDH problem, we can use it to solve the CDH problem.
    
    Hint: $(a + b)^2 = \dots$
  \end{enumerate}

  You can assume that finding the square root in $\GG$ is efficient.

  \begin{proof}
    If $q = 2$, the group has two elements and finding the square root is trivial.
    Suppose then that $q$ is an odd prime.
    Since $\GG$ is cyclic, it follows from Lagrange that $g^q = 1$, and so $g^{q+1} = g$.
    Because $q$ is odd, then $q+1$ is even, and so $(q+1)/2$ is an integer.
    We can thus efficiently calculate $h \gets g^{(q+1)/2}$ since division and exponentiation are efficient, and $h$ is a square root of $g$.
  \end{proof}
\end{task}

\begin{task}{Insecurity of ECB}
  We have seen in class that the ECB block cipher mode of operation is insecure.
  Indeed, think about the encryption of the picture of Tux the penguin in ECB mode.

  Formalise the weakness of ECB by building a distinguisher which wins the CPA game, i.e. show that ECB mode is not IND-CPA.
\end{task}

\begin{task}{A costly estimation}
  The goal is to construct a $2^{128}$-secure digital signature scheme $s$ from a one-way function $f$.
  For that, cryptographers presented a black-box construction $s=P^f$ and a security proof with a statement:
  \begin{quote}
    For any adversary $\AD$ that breaks $s$ with running time $t$ and success probability $\delta$ so that $t/\delta\le S$, there exists an adversary with running time $t^2$ that breaks $f$ with probability $\delta^2$.
  \end{quote}
  How secure does $f$ have to be for that cryptographic construction?
\end{task}

\begin{task}{Baby-step giant-step}
  % https://www.mat.uniroma2.it/~geatti/HCMC2023/Lecture4.pdf
  The baby-step giant-step algorithm is a deterministic algorithm for finding the discrete logarithm in a cyclic group.
  It exploits the fact that every element $h \in \GG$ can be written as
  \[
    h = g^{r + mk}
  \]
  for integers $r, m, k$ satisfying $m = \bigl\lceil\sqrt{q}\bigr\rceil$, and $0 \le k, r < m$, which we can rewrite as
  \[
    g^{r} = hg^{-mk}.
  \]

  We can then find the logarithm $x = \log_g(h)$ by comparing two lists: the baby steps $g^r$ and the giant steps $hg^{-mk}$, for $0 \le k, r < m$, where $g^{-m}$ can be precomputed.
  Indeed, if we find a coincidence between the two lists, i.e. we find $k_0, r_0$ such that $a^{r_0} = hg^{-mk_0}$, then
  \[
    \log_g{h} = r_0 + mk_0.
  \]
  Notice that computing both lists takes at most $2m \sim 2 \sqrt{q}$ steps.
  The full running time then depends on the efficiency of list comparison.

  \renewcommand{\algorithmicrequire}{\textbf{Input:}}
  \renewcommand{\algorithmicensure}{\textbf{Output:}}
  \begin{algorithm}
    \caption{Baby-step giant-step (BSGS) algorithm}\label{alg:cap}
  \begin{algorithmic}[1]
    \Require $\GG$ of order $q$, a generator $g$, and some element $h\in\GG$ for which to find the DL.
    \Ensure A value $x$ satisfying $g^x = h$.
    \State $m \gets \bigl\lceil \sqrt{q}\bigr\rceil$
    \State $\gamma \gets g^{-m}$
    \State $B \gets \{(g^r,r)\}$ for $r = 0, 1, \dots, m - 1$
    \State $G \gets \{(h\gamma^k,k)\}$ for $k = 0, 1, \dots, m - 1$
    \State Find $(\beta, r)\in B$ and $(\beta', k)\in G$ where $\beta = \beta'$
    \State $x \gets (km + r) \bmod q$
    \State \Return{$x$}
  \end{algorithmic}
  \end{algorithm}
\end{task}

Find the discrete logarithm of $h=9$ in the group generated by $g = 4$ in $\ZZ_{23}$ using the BSGS algorithm.

\newpage

\begin{center}
  Concept refresher
\end{center}

\begin{tcolorbox}[title=Computational Diffie-Hellman (CDH) assumption]
  A group $\GG$ of order $q$ is a secure CDH-group iff for any efficient adversary $\AD$, the adversary cannot win the following game with probability better (or worse) than $1/2$.
  \begin{center}
    \begin{tikzpicture}[
      Squiggly/.style={
        decorate,
        decoration={snake, segment length=4, amplitude=0.9, pre length=.38ex},
      },
      ]
      \draw[rounded corners,thick,fill=blue!50] (0, 0) rectangle  ++(3,-3);
      \draw[rounded corners,thick,fill=red!85] (4.5, 0) rectangle ++(1,-3);
    
      \node (P) at (1.5, -0.5) {$\text{Challenger } \GAME$};
      \node (K) at (5, -0.5) {$\AD$};

      \node at (1.5, -1.75) {
        $\begin{aligned}
          x &\gets \ZZ_q\\
          y &\gets \ZZ_q\\
          z &\iseq xy
        \end{aligned}$};
    
      \draw[->,thick] (3, -0.5) -- (4.5, -0.5) node[midway,above] {$g$};
      \draw[->,thick] (3, -1.25) -- (4.5, -1.25) node[midway,above] {$g^x, g^y$};
      \draw[<-,thick] (3, -2.5) decorate[Squiggly]{-- (4.5, -2.5) node[midway,above] {$z$}};
    \end{tikzpicture}
  \end{center}
\end{tcolorbox}

\begin{tcolorbox}[title=Decisional Diffie-Hellman (DDH) assumption]
  A group $\GG$ of order $q$ is a secure DDH-group iff for any efficient adversary $\AD$, when given the following two games, the adversary cannot determine the correct scenario with probability better (or worse) than $1/2$.
  \begin{center}
    \begin{tikzpicture}[
      Squiggly/.style={
        decorate,
        decoration={snake, segment length=4, amplitude=0.9},
      },
      ]
      \draw[rounded corners,thick,
        fill=blue!50
      % pattern=crosshatch dots,
      % blur shadow={
      %   fill=blue!50,
      %   shadow opacity=40,
      %   shadow blur radius=0.8ex,
      %   shadow blur steps=20,
      %   shadow xshift=0,
      %   shadow yshift=-.2ex,
      %   shadow scale=1}
      ] (0, 0) rectangle  ++(3,-3);
      \draw[rounded corners,thick,fill=red!85] (4.5, 0) rectangle ++(1,-3);

      \draw[rounded corners,thick,fill=red!85] (6, 0) rectangle ++(1,-3);
      \draw[rounded corners,thick,fill=blue!50] (8.5, 0) rectangle  ++(3,-3);
    
      \node (P) at (1.5, -0.5) {$\text{Challenger } \GAME_0$};
      \node (K) at (5, -0.5) {$\AD$};

      \node (K) at (6.5, -0.5) {$\AD$};
      \node (P) at (10, -0.5) {$\text{Challenger } \GAME_1$};

      \node at (1.5, -1.75) {
        $\begin{aligned}
          x &\gets \ZZ_q\\
          y &\gets \ZZ_q\\
          z &\gets \ZZ_q
        \end{aligned}$};

      \node at (10, -1.75) {
        $\begin{aligned}
          x &\gets \ZZ_q\\
          y &\gets \ZZ_q\\
          z &\gets xy
        \end{aligned}$};
    
      \draw[->,thick] (3, -1) -- (4.5, -1) node[midway,above] {$g$};
      \draw[->,thick] (3, -1.75) -- (4.5, -1.75) node[midway,above] {$g^x, g^y$};
      \draw[->,thick] (3, -2.5) -- (4.5, -2.5) node[midway,above] {$g^z$};

      \draw[<-,thick] (7, -1) -- (8.5, -1) node[midway,above] {$g$};
      \draw[<-,thick] (7, -1.75) -- (8.5, -1.75) node[midway,above] {$g^x, g^y$};
      \draw[<-,thick] (7, -2.5) -- (8.5, -2.5) node[midway,above] {$g^z$};
    
      \draw[->,thick] (5,-3) decorate[Squiggly]{ -- (5, -3.36) };
      \draw[->,thick] (6.5,-3) decorate[Squiggly]{ -- (6.5, -3.36) };
    
      \node at (5, -3.6) {$b$};
      \node at (6.5, -3.6) {$b$};
    \end{tikzpicture}
  \end{center}
\end{tcolorbox}

\end{document}
