\documentclass{practice}

\usepackage{amsthm}
\usepackage{crysymb}

\usepackage{tikz}
\usetikzlibrary{calc,decorations.pathmorphing,shapes,positioning,patterns,shadows.blur}

\usepackage{algorithm}
\usepackage{algpseudocode}
\newcommand*{\algorithmautorefname}{Algorithm}
\renewcommand{\algorithmiccomment}[1]{\hfill$\smalltriangleright$\textit{#1}}

\title{8}
\subtitle{RSA cryptosystem}
\date{\DTMdate{2024-10-30}}

\begin{document}
\maketitle

\begin{task}{Pen and paper}
  Given primes $p = 5$, $q = 11$, and the public exponent $e = 3$, find the private exponent $d$.

  Then, encrypt $m = 6$, and decrypt the obtained ciphertext.
  Finally, decrypt $c = 9$.
\end{task}

\begin{task}{Picky parameter}
  If $p = 7$ and $q = 11$, can you still use $e = 3$ as a public exponent?
  If not, what is the smallest exponent you can use?
\end{task}

\begin{task}{The cost of speed}
  Suppose that $p, q$ are 3072-bit primes, and $e = 3$ is the RSA public exponent.
  A small public exponent helps speed up encryption.

  Encrypt $m = 5$.
  Is confidentiality maintained against a computational adversary who cannot factor $pq$?

  \begin{tcolorbox}[title={$e=65537$}]
    $e = 65537 = 2^{2^4} + 1$ is commonly used as the public exponent for RSA, even in modern implementations where the message is properly padded.
  \end{tcolorbox}
\end{task}

\begin{task}{IND-CPA? Never heard of her\dots}
  Win the single query IND-CPA game against textbook RSA.
\end{task}

\begin{task}{The textbook is a lie}
  Win the single query IND-CCA2 game against textbook RSA.

  Hint: textbook RSA is malleable.
\end{task}

\begin{tcolorbox}[title=RSA-OAEP]
  In practice, RSA encryption should be used with the Optimal Asymmetric Encryption Padding (OAEP) padding scheme.
  OAEP makes RSA encryption nondeterministic and prevents various attacks, some of which you will see next week.
\end{tcolorbox}

\newpage

\begin{task}{In search of unity}
  How many square roots of unity are there in $\ZZ_{21}$?
  Find them.
\end{task}

\begin{task}{Miller-Rabin}
  The Miller-Rabin primality test is given in \autoref{alg:mr}.

  \renewcommand{\algorithmicrequire}{\textbf{Input:}}
  \renewcommand{\algorithmicensure}{\textbf{Output:}}
  \begin{algorithm}
    \caption{Single round Miller-Rabin test}\label{alg:mr}
  \begin{algorithmic}[1]
    \Require $n > 2$, an odd integer to be tested for primality.
    \Ensure $0$ if $n$ is found to be composite, $1$ otherwise.
    \State $a \getsu \{2, \dots, n - 2\}$
    \Comment {$a$ is called a \emph{witness} if it helps show that $n$ is composite.}
    \If{$\gcd(a,n)\neq1$} 
      \Return{$0$}
    \EndIf
    \State Find $k, m$ such that $(n-1)=2^k \cdot m$ for an odd $m$ 
    \If{$a^m \bmod n =1 $}
      \Return{$1$}
    \EndIf
    \If{$a^{m\cdot2^{i}} \bmod n = n - 1$ for an $i = 1, \dots, k-1$}
      \Return{$1$}
    \EndIf
    \State \Return{$0$}
  \end{algorithmic}
  \end{algorithm}

  Note that if the algorithm outputs $1$, it means that the number might be prime (probably prime), not that it is prime.
  If the algorithm outputs $0$, the number is certainly composite.
  In practice, the algorithm must be repeated multiple times, with the number of rounds depending on the required security level.

  % https://oeis.org/A020233
  Use the Miller-Rabin test to verify whether $25$ is prime with potential witnesses $7, 6$.
  Then, verify whether $19$ is prime with potential witnesses $4, 5$.
\end{task}

\newpage

\begin{center}
  Concept refresher
\end{center}

\begin{tcolorbox}[title=Euler's totient (phi) function]
  Euler's \emph{totient} function $\varphi(n)$ counts the positive integers up to a given integer $n$ that are coprime to $n$.
  One way to compute $\varphi(n)$ is with the prime factorisation of $n$.
  Euler's totient function is then computed as
  \[
    \varphi(n) = p_1^{k_1 - 1}(p_1 - 1) p_2^{k_2 - 1}(p_2 - 1) \dots p_r^{k_r - 1}(p_r - 1)
  \]
  where $n = p_1^{k_1}p_2^{k_2}\dots p_r^{k_r}$ is the prime factorisation of $n$.

  Note that:
  \begin{itemize}
    \item If $\gcd(m, n) = 1, \varphi(m\cdot n) = \varphi(m)\cdot\varphi(n)$
    \item For a prime $p$, $\varphi\bigl(p^k\bigr) = p^{k-1}(p-1)$
  \end{itemize}
  \tcblower
  If $n$ is the product of $i$ distinct primes, $\varphi(n) = (p_1 - 1)(p_2 - 1)\cdots(p_i - 1)$.
\end{tcolorbox}

Given large enough numbers with no small factors, integer factorisation is an intractable problem for limited, classical adversaries.
It follows that computing $\varphi$ is also intractable under the same circumstances.
When the integer factorisation of $n$ is known, $\varphi(n)$ is easily computed, making it suitable for a cryptographic trapdoor function where the factorisation is the trapdoor.

Integer factorisation is not intractable for quantum adversaries due to Shor's algorithm.
As such, RSA is not suitable for post-quantum cryptography.
Integer factorisation is not intractable for unlimited adversaries who can simply brute force the solution.

\end{document}
