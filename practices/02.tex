\documentclass{practice}

\title{2}
\subtitle{Unbreakable ciphers}
\date{\DTMdate{2024-09-18}}

\DeclareMathOperator*{\IC}{IC}
\DeclareMathOperator*{\mIC}{\IC_M}

\begin{document}
\maketitle

\begin{task}{Identifying ciphers}
  Given the ciphertext \texttt{JPOWLY} and the plaintexts
  \begin{itemize}
    \item \texttt{CIPHER}
    \item \texttt{ENCDEC}
    \item \texttt{SHIFT}
    \item \texttt{TRYSUB}
  \end{itemize}
  determine for each plaintext whether the corresponding ciphertext could have been obtained with: the shift cipher, the simple substitution cipher, the one time pad.
  If not, explain why.
\end{task}

\begin{task}{A fair coin\dots}
  Suppose that you toss a fair coin twice.
  Let $X$ be defined as the number of heads you observe.

  Find the range of $X$ as well as its probability mass function.
\end{task}

\begin{task}{\dots{}but biased dice}
  Suppose now that you roll two dice, one of which is balanced, but the other one is weighted
  so as to give the probabilities:
  \begin{table}[h!]
    \centering
    \begin{tabular}{@{}cccccc@{}}
      1 & 2 & 3 & 4 & 5 & 6\\\midrule
      $1/9$ & $2/9$ & $1/9$ & $2/9$ & $1/9$ & $2/9$
    \end{tabular}
  \end{table}
  \vspace{-0.7em}

  Let $X$ be defined as the sum of the face values.
  Find the range of $X$ as well as its probability mass function.
\end{task}

\begin{task}{Conditional probability}
  Let $A$, $B$ be two events, each with a probability $\frac{1}{3}$ of happening.
  With probability $\frac{1}{3}$, neither event happens.

  What is the probability $\Pr[A\vert B]$, i.e. that $A$ happens knowing that $B$ happened.
\end{task}

\begin{task}{XOR}
  Write out the truth table for the XOR binary operation, which we write as $\oplus$.
  Then, answer the following:
  \begin{itemize}
    \item Is XOR associative?
    \item Is XOR commutative?
    \item What is the inverse of an element under XOR?
    \item What is the identity element under XOR?
  \end{itemize}
\end{task}

\begin{task}{Identify the message}
  Given the ciphertext $c$ encrypted with the key $k$ using OTP with
  \begin{align*}
    c &= \texttt{11100011 00011101 10000110}\\
    k &= \texttt{10101100 01001001 11010110}
  \end{align*}
  find the plaintext message.
  Use the ASCII table given in \autoref{table:asciiupper} to convert it to a meaningful word.

  \begin{table}[h!]
    \centering
    \begin{tabular}{@{}lcccccccccccccc@{}}
      Letter & A  & B  & C  & D  & E  & F  & G  & H  & I  & J  & K  & L  & M & \dots\\\midrule
      DEC & 65 & 66 & 67 & 68 & 69 & 70 & 71 & 72 & 73 & 74 & 75 & 76 & 77 & \dots\\
      HEX & 41 & 42 & 43 & 44 & 45 & 46 & 47 & 48 & 49 & 4A & 4B & 4C & 4D & \dots\\\\
      Letter & \dots & N  & O  & P  & Q  & R  & S  & T  & U  & V  & W  & X  & Y  & Z\\\midrule
      DEC & \dots & 78 & 79 & 80 & 81 & 82 & 83 & 84 & 85 & 86 & 87 & 88 & 89 & 90\\
      HEX & \dots & 4E & 4F & 50 & 51 & 52 & 53 & 54 & 55 & 56 & 57 & 58 & 59 & 5A
    \end{tabular}
    \caption{ASCII character codes of capital English letters.}
    \label{table:asciiupper}
  \end{table}
\end{task}

\begin{task}{A lack of information}
  Given the ciphertext $c = \texttt{10111010 01001101 00010101}$ encrypted with the OTP using one of the following keys recovered via brute-force:
  \begin{align*}
    k_1 &= \texttt{11111001 00001100 01000001}\\
    k_2 &= \texttt{11111110 00000010 01010010}\\
    k_3 &= \texttt{11110101 00011010 01011001}
  \end{align*}
  determine whether I am a cat, a dog, or an owl person.

  If the ciphertext was obtained via a monoalphabetic substitution, could you then be able to tell the correct answer?
\end{task}

\begin{task}{Many-time-pad}
  I just learnt about the OTP and want to exchange a message with my friend.
  However, because I am now a professional cryptographer after last week's class, I have decided to optimise the OTP:
  I use a key that's half the size of the message, and use the key on both halves similar to Vigenere's key repetition.

  Suppose that Eve the eavesdropper sees my sent ciphertext.
  Can she possibly decrypt the message?
  After all, the OTP is unconditionally secure\dots{} right?
  
  What is the type of attack called that Eve must perform (e.g. KPA, CPA, \dots)?
  Is it active or passive?
\end{task}

\begin{task}{Unicity distance}
  Compute the unicity distance of the following ciphers:
  \begin{itemize}
    \item Caesar's cipher
    \item Affine cipher
    \item Vigenere cipher (in terms of key-length $n$)
  \end{itemize}
\end{task}

\begin{task}{Malleability}
  Consider the ASCII string \enquote{RAND}, a random key $k$, and the OTP ciphertext $c$, with:
  \begin{align*}
    m &= \texttt{01010010 01000001 01001110 01000100}\\
    k &= \texttt{01000011 01011000 11011011 10100101}\\
    c &= \texttt{00010001 00011001 10010101 11100001}
  \end{align*}

  Suppose now that Eve flips the \emph{least significant bit} of the first byte of the ciphertext, i.e. the rightmost bit of the first ciphertext block of bits.
  Eve then forwards the message to Bob who thinks that the message comes from Alice.
  What message will Bob decrypt?
  What problems do you think arise from the malleability of a cipher?
\end{task}


\end{document}
