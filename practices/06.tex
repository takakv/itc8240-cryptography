\documentclass{practice}

\usepackage[dvipsnames]{xcolor}

\title{6}
\subtitle{Public key cryptography, key establishment, and abstract algebra}
\date{\DTMdate{2024-10-16}}

\usetikzlibrary{positioning,calc}

\usepackage{listings}

\begin{document}
\maketitle

If not specified otherwise, we treat $\GG$ as a multiplicative group, i.e. the group operation is multiplication.

\begin{task}{Hello proof}
  Show that the identity element in a group is unique.
\end{task}

\begin{task}{Confusing terminology}
  What is the order of a group?
  What is the order of an element of a finite group?

  Let $g = 3$ be an element in $\ZZ_{11}$.
  \begin{itemize}
    \item What is the order of $\ZZ_{11}$?
    \item What is the order of $g$ in $\ZZ_{11}$?
    \item Does $g$ generate $\ZZ_{11}$?
  \end{itemize}
\end{task}

\begin{task}{Going in circles}
  % https://feog.github.io/grpex.pdf
  Let $\GG = \ZZ^*_{20}$ be the group of invertible elements in $\ZZ_{20}$.
  Find two subgroups of order $4$ in $\GG$, one that is cyclic and one that is not cyclic.
\end{task}

\begin{task}{Learning by analogy}
  You have two padlocks, one for which you have two keys, the other for which you have one key.
  You also have two boxes that can be locked with padlocks: a regular box and a letterbox.

  Explain symmetric and public key encryption using these elements.
\end{task}

\begin{task}{Eavesdropping on Diffie-Hellman}
  Draw the Diffie-Hellman key exchange, and write out what Alice and Bob get at the end of the exchange.
  Write out what the passive eavesdropper Eve gets at the end of the exchange.

  What equation does Eve need to be able to solve to recover the shared secret?
\end{task}

\begin{task}{Establishing a shared secret}
  Write out the Diffie-Hellman key exchange in $\ZZ_{23}$ with the generator $g = 4$.

  Compute what Alice, Bob, and Eve get.
\end{task}

\newpage

\begin{task}{Active attack}
  Eve has an older brother Mallory.
  Mallory is a seasoned hacker and can perform man-in-the-middle attacks, i.e. he is an active adversary.

  Show how Mallory can attack the Diffie-Hellman key-exchange in an unauthenticated channel.
  Can Alice and Bob notice the attack over the same channel?
\end{task}

\begin{task}{Think different}
  Suppose now that instead of doing Diffie-Hellman in $(\ZZ^*_q, \cdot)$, we do it in $(\ZZ_q, +)$.
  
  Write down the DLP in an additive group.
  Does this still seem infeasible to solve for large values of $q$?
\end{task}

\newpage

\begin{center}
  Concept refresher
\end{center}

\begin{tcolorbox}[title=Definition of a group]
  A group $(\GG, \circ)$ is a set $\GG$ together with a binary operation
  \[
    (a, b)\to a\circ b: \GG \times \GG \to \GG
  \]
  satisfying the following conditions:
  \begin{itemize}
    \item (associativity) for all $a,b,c\in\GG$,
    \[
      (a \circ b) \circ c = a \circ (b\circ c);
    \]
    \item (existence of the \emph{identity element}) there exists $e\in\GG$ s.t.
    \[a \circ e = a = e \circ a\]
    for all $a \in \GG$;
    \item (existence of inverses) for each $a\in\GG$, there exists $a^{-1}$ s.t.
    \[
      a\circ a^{-1} = e = a^{-1} \circ a.
    \]
  \end{itemize}
\end{tcolorbox}

\begin{tcolorbox}[title=Discrete logarithm problem]
  Let $\GG = \langle g \rangle$ be a $q$-element multiplicative group generated by $g$.
  The discrete logarithm is defined as the smallest integer $x < q$ such that
  \[
    h = g^x
  \]
  and is \emph{intractable} if an adversary cannot find $x$ with probability better than $1/q$.
  \tcblower
  Let $q$ be a prime such that $p = 2q + 1$ is prime.
  We call such a prime $p$ a \emph{safe prime}.
  Then there exists $\GG = \{g^i : 0 \le i < q\}$ such that $\lvert \GG \rvert = q$.
  In this group, we define the discrete logarithm problem as
  \[
    \forall h\in\GG : \log(h) = x \Leftrightarrow g^x \equiv h \pmod{p}
  \]
  and it is intractable for large enough $q$, e.g. for $q$ larger than $3072$ bits.
\end{tcolorbox}

\end{document}
