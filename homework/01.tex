\documentclass{homework}

\usepackage{fancyvrb}
\fvset{listparameters=\setlength{\topsep}{0pt}\setlength{\partopsep}{0pt}}

\title{01}
\author{Taaniel Kraavi}
\date{\DTMdate{2024-10-18}}

\begin{document}
\maketitle

\textbf{Organisational details}
\begin{itemize}
  \item You can submit assignments up to 1 week late, however, late assignments incur a 50\% grade penalty.
  After the 1 week grace period the assignment's grade will be 0.
  \item Your submission must be a PDF.
  You can write solutions by hand and scan/photograph them, or write them using \LaTeX{} (recommended).
  If you use MS Word, you must use the equation editor for typesetting the mathematics.
  \begin{itemize}
    \item If you write solutions by hand, please make sure that the scans/pictures are of good quality (high resolution, good contrast, no blur).
    \item If your submission is not legible or is otherwise of poor quality (e.g. crossed out solutions, scribbles, \dots) we will not grade the impacted task(s).
    \item If you write code to help you solve tasks, add the code file(s) to your Moodle submission or push the code to a \emph{private} GitHub repository.
    Add \texttt{takakv} as a collaborator.
  \end{itemize}
  \item You must explicitly write out the reasoning for your solutions.
  Providing an unexplained answer will give you no points.
  \begin{itemize}
    \item Brute force is not a solution!
    \item If you write code, you must comment your code and explain what it does and why are you doing things in that way.
  \end{itemize}
\end{itemize}

\vspace*{1em}

\textbf{About help}
\begin{itemize}
  \item Homework assignments are individual work.
  While you may discuss the tasks with one-another and give hints or suggest resources, all submissions must be your own work, stem from your own thought process, and be written in your own words.
  You are not allowed to share answers, even for cross-checking.
  \item You may not use third party/online tools to solve tasks, but you may use them to check your tasks.
  \item If you use external resources\footnote{Anything that is not part of lecture or practice slides.} to help you with your homework, e.g. a CSE\footnote{Cryptography Stack Exchange} post or an article, you must properly cite that resource.
  \item If we catch you using someone else's work, we will report you for plagiarism.
  If we catch you colluding with a coursemate, we will report both of you for plagiarism.
  In both cases you will fail the course.
\end{itemize}

\newpage

\begin{task}{Locksmith}
  Suppose that you have intercepted the ciphertext \texttt{0xCA0BEC266D6D0C} that was obtained with the one time pad.
  Find the keys such that the ciphertext would decrypt to the words:
  \begin{itemize}
    \item correct
    \item mistake
    \item perhaps
  \end{itemize}

  \textit{(1 point)}
\end{task}

\begin{task}{An affinity for animals}
  Given the alphabet
  \[
    \mathcal{A} = \{a, b, c, d, e, f, g, h, i, k, l, m, n, o, t, u, w, x, y\},
  \]
  the cipher defined as
  \begin{align*}
    \Enc_{(a, b)}(x) &= (ax - b) \bmod \lvert\mathcal{A}\rvert,\\
    \Dec_{(a,b)}(y) &= a^{-1}(y + b) \bmod \lvert\mathcal{A}\rvert,
  \end{align*}
  and the plaintext-ciphertext pair $(\mathit{dog},\mathit{hdc})$,
  find:
  \begin{enumerate}
    \item the unicity distance of the particular cipher instance, \textit{(1 point)}
    \item the animal corresponding to the ciphertext $\mathit{wdf}$. \textit{(2 points)}
  \end{enumerate}
\end{task}

\begin{task}{Classic Vigenere}
  Given the ciphertext
  \begin{Verbatim}
LWMVWAHAMEHAGRGOPGXGAVVSJWEZMNSRGFWUPCWWSRQXARTVVQKUZIWVDUW
SJTQRLWGLIHWCLIFLIWWMJGMRVWGGSMJDERHJXDEUQXAVWSATCLGHCVJWCL
IJWKMVQGCMWQGJUMYZIKLGGHMXGYXDIALJXSMLDNGGFKMRAWCKIGJJVHWJI
PIHGECPSJEZILWMBXZSIXVANPKCAKDVPQJTYYAJTLFQLWWWWOWWLSNTASEW
IPMFYIWLAVTJYLKPGMFYIPELQDCHGFIVIWVDZASFIXVANPKCTWRIYKWNWYZ
SKMRGLWQRYLDPMVWXAXGSHAYEWIPELFDWRWKWWYDVWIZWGGKSMDSPENWIWL
AVTIRQLWQRYACKPMVXVKLZTQVAEBQKJSIQSFKIIXMKJVIEHAWCEWCBLAKIW
VQXXVEFUXIPZAHBSJQPVHZWPTXZJTKSJVHGSMJTIWKMBQRYLWIXFGDVIAFR
TYVACOCGMGAIDXBQKZLDJNWUIBSJWKMEDACOXGSCGSFWXVJGJBIXAGCIFGM
IBLWAGZIDAVQSMKQMPAWUATGDXBMUSAIJXAAQELADVWSFSAIPMPTEULXDML
ATAEKUPAYSDAGEKKDUIUZDWWWLDZINWPTXZWXZQGNXMEFVBCWAUIIWLWHIR
VJTIHAFVXVWXTZIFUTAYDLXUELWAGWSQXVKLZPBCGMSWRLUPZISTDCXHJXD
EUQQMGSMHMCGMWIZWFDBLAFVBSZASMMKFDLMXXTZIFLUZSEKPGMFYNWYVGC
BGSJTIFGMINVWWSWQGXHXIWUWJIUSJAIQGJPENWCWXZACOXGKPGSJLWIXQG
JLSFLRIVWSQWYLXGMIVGBWJLZTXVWKHJIUSJAIQGJLSFLAQOWLDZISVDZXZ
SIGSMVDVXUSGMETGJBJJWTLSEGUZIDAVQSFTTKEMKTGSMVDVXTWAQINWXVK
GVDZXZSIGSMVDVXUSGMETGJBXZWUZIWVDUXGHTIGWSQTCSKHMQTDTJIUSJA
IQGJZISDPHCSFIQWGUXIPSYDZEHZDJIBMHBFWUPCWWLWQWGJIPELXGMIVGB
UMYZIVSLZPDIEWPVMFYIWCGMIWHSQSWIKFIUISFIPELLWIXALSWIKFIWVOG
CBLSNTUISFXVKLGBWVJGLBSQGJWVLGNWYJFTQKZTDZSJLDBLWUGWAVKDNTJ
ACKMHDTLHAKHQHWFIAMOSHNSDDDEMFYDVQQHWWRWOWWAWJTXVGLTAXAFVPE
DXLICSUGWWKLWMTDSCMXZGEQRYLDOEAFYCWLSUZEULXWRGXIPIXJTMHGEIP
ELENKSMFIZCOSHJYKAAGHAKBIRLDXVK
  \end{Verbatim}
  encrypted with Vigenere:
  \begin{enumerate}
    \item Find the key-length by Kasiski or by the IC method. \textit{(2 points)}
    \item Find the key. \textit{(2 points)}
    \item Decrypt the message. \textit{(1 point)}
  \end{enumerate}

  You are allowed to use third party and online tools to help you confirm your solutions, but not to simply solve the task.
  You must show your reasoning for each step, and reference any online tool you have used for checking, if any.
  If you write a program, you can write your reasoning as code comments.
\end{task}

\begin{task}{PRFs}
  Let $F_k: \{0,1\}^\lambda \to \{0,1\}^\lambda$ be a secure PRF.
  Are the following functions PRFs?
  If not, provide a distinguisher.
  If yes, attempt to explain why you think so (no need for a formal proof).

  \begin{enumerate}
    \item $F'_k(x \vert\vert x') = F_k(0\vert\vert x) \oplus F_k(x'\vert\vert 1)$, where $x$ and $x'$ are each $\lambda - 1$ bits long. \textit{(1 point)}
    \item $F'_k(x \vert\vert x') = F_k(x \oplus 0^\lambda) \oplus F_k(x' \oplus 1^\lambda)$, where $x$ and $x'$ are each $\lambda$ bits long. \textit{(1 point)}
    \item $F'_k(x \vert\vert x') = F_k(x) \oplus x'$, where $x$ and $x'$ are each $\lambda$ bits long. \textit{(1 point)}
  \end{enumerate}
\end{task}

\begin{task}{A malleable stream}
  You have have intercepted a message encrypted with a (synchronous) stream cipher that Alice has sent Bob.
  You suspect that the message is about Bob asking Alice out, and that Alice has said \enquote{yes}.
  However, you think that Alice's poor choice of cryptography (namely using an unauthenticated cipher) makes her a poor fit for Bob.
  Thus, you want to modify the message such that Bob decrypts it to \enquote{no!}.

  The ciphertext you intercepted is \texttt{0x3FC6A9}.
  What ciphertext would you send Bob such that he would decrypt it to \enquote{no!}?
  Assume that the plaintext encoding is ASCII. \textit{(3 points)}
\end{task}

\newpage

\begin{task}{CBC-CTR (bonus)}
  The (in)famous cryptographer Dr. Cry has devised a new block cipher mode of operation called CBC-CTR.
  In CBC-CTR, the IV is picked at random for the first message sent, just like for regular CBC.
  However, for every subsequent message sent (entire message, not message block) the initial IV is incremented by one.

  That is, the first message is encrypted with the IV $iv$, the second one with the IV $iv + 1$, etc\dots.

  Show that this scheme does not have IND-CPA. \textit{(2 points)}
\end{task}
\end{document}
