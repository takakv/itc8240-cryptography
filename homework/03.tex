\documentclass{homework}
\newcommand*{\MAC}{\mathsf{MAC}}

\title{03}
\date{\DTMdate{2024-12-06}}

\begin{document}
\maketitle

\textbf{Organisational details}
\begin{itemize}
  \item You can submit assignments up to 1 week late, however, late assignments incur a 50\% grade penalty.
  After the 1 week grace period the assignment's grade will be 0.
  \item Your submission must be a PDF.
  You can write solutions by hand and scan/photograph them, or write them using \LaTeX{} (recommended).
  If you use MS Word, you must use the equation editor for typesetting the mathematics.
  \begin{itemize}
    \item If you write solutions by hand, please make sure that the scans/pictures are of good quality (high resolution, good contrast, no blur).
    \item If your submission is not legible or is otherwise of poor quality (e.g. crossed out solutions, scribbles, \dots) we will not grade the impacted task(s).
    \item If you write code to help you solve tasks, add the code file(s) to your Moodle submission or push the code to a \emph{private} GitHub repository.
    Add \texttt{takakv} as a collaborator.
  \end{itemize}
  \item You must explicitly write out the reasoning for your solutions.
  Providing an unexplained answer will give you no points.
  \begin{itemize}
    \item Brute force is not a solution!
    \item If you write code, you must comment your code and explain what it does and why are you doing things in that way.
  \end{itemize}
\end{itemize}

\vspace*{1em}

\textbf{About help}
\begin{itemize}
  \item Homework assignments are individual work.
  While you may discuss the tasks with one-another and give hints or suggest resources, all submissions must be your own work, stem from your own thought process, and be written in your own words.
  You are not allowed to share answers, even for cross-checking.
  \item You may not use third party/online tools to solve tasks, but you may use them to check your tasks.
  \item If you use external resources\footnote{Anything that is not part of lecture or practice slides.} to help you with your homework, e.g. a CSE\footnote{Cryptography Stack Exchange} post or an article, you must properly cite that resource.
  \item If we catch you using someone else's work, we will report you for plagiarism.
  If we catch you colluding with a coursemate, we will report both of you for plagiarism.
  In both cases you will fail the course.
\end{itemize}

\newpage

\begin{task}{Hybrid gaming}
  A hybrid encryption scheme uses public key cryptography for the key exchange/establishment, but symmetric cryptography for subsequent message encryption (and authentication).

  Let $g$ be the generator of a group $\GG$ with a prime number $q$ of elements and where the DLP is hard.
  Suppose that Alice and Bob know each-other's ElGamal public keys.
  
  \begin{enumerate}
    \item Show how Alice and Bob can compute a shared secret $x$ without exchanging any messages. \textit{(1 point)}
    \item Suppose now that Alice and Bob use $H(x)$ as an AES secret key, where $H$ is a key-derivation function which derives a $128$-bit key from $x$.
    Subsequent messages exchanged by Bob and Alice are encrypted with AES-GCM.

    Is this protocol IND-CCA2 secure? (\textit{1 point})
  \end{enumerate}
\end{task}

\begin{task}{I am the senate}
  You are a political spy for the duchy of Mallea who has infiltrated the senate of the republic of Cecea.
  The senate of Cecea is holding emergency elections to pass a bill mandating the use of IND-CCA2 cryptosystems.
  This would be bad news for Mallea, so your job is to prevent the bill from passing.

  Let $g$ be the generator of a group $\GG$ with a prime number $q$ of elements and where the DLP is hard.
  The elections are conducted with the lifted ElGamal cryptosystem, i.e. ElGamal where $g^m$ is encrypted instead of the message $m$.
  No votes are individually decrypted.
  Instead, all the encrypted votes are multiplied together and only the end result is decrypted.
  This maintains voting secrecy.
  This means that all those in favour of the new bill must encrypt $g^1$ and those against it must encrypt $g^0$.

  The senate has 101 members, you included, and each senator has one vote.
  Therefore, the bill passes if the end result is at least $g^{51}$.
  Your intel tells you that at least 70 senators will vote in favour of the bill, and at least $14$ senators will vote against the bill.
  This information is not known to anyone else.

  How do you manipulate the election such that the bill does not pass?
  You are not able to influence the decision of other senators or swap/drop their ballots.
  
  \textit{(2 points)}
\end{task}

\newpage

\begin{task}{A shared problem}
  Let the hash function $H_n$ be defined as
  \[
    H_n: \ZZ_n \times \ZZ_n \to \ZZ_n;\; (a, b) \to (ab) \bmod{n}.
  \]

  % d = 27
  Let the RSA public key of an access control system (ACS) be $(55, 3)$.
  The ACS gives access to anyone who can present a valid access token---an unordered pair of integers---signed by the ACS and which has not already been used.
  That is, each token can only be used once.
  Furthermore, suppose that factoring $55$ is hard, i.e. you cannot find the private key.

  The ACS issues a textbook RSA signature over the hash of the token computed with $H_{55}$.
  That is, to sign the token $(a, b)$, the ACS computes $\sigma \gets [H_{55}(a,b)]^d \bmod{55}$ where $d$ is the private key of the ACS.

  Using your hacking skills, you are able to sniff which valid token-signature pairs have been previously presented to the ACS.
  You recover the following token-signature pairs:
  $[(6, 15), 40], [(7, 16), 18], [(8, 17), 16]$.

  How can you gain access with the token $(4, 28)$?
  Explain your approach and present the token-signature pair.
  \textit{(2 points)}
\end{task}

\begin{task}{Uninvited guest}
  % p1 = 3
  % q1 = 29
  % p2 = 11
  % q2 = 17
  % p3 = 5
  % q3 = 23

  % m = 7
  Alice wants to meet her cryptographer friends and decides to send an invitation with the time and place (encoded into an integer) to her three friends.
  The invitation sent is the same and is encrypted with textbook RSA.
  The RSA public keys of her friends are $\pk_1 = (87, 3), \pk_2 = (187, 3), \pk_3 = (115,3)$.

  Eve intercepts the invitations $c_1 = 82, c_2 = 156, c_3 = 113$.
  Assuming that Eve cannot recover the secret keys from the public keys, how can Eve recover the original invitation/message?
  What is the message?

  Show your solution process/calculations.
  You are allowed to use a calculator to compute: multiplication, modulo, and multiplicative inverses.

  \textit{(2 points)}
\end{task}

\newpage

\begin{task}{Who cracks first?}
  Let $\GG$ be a group of prime order $q$ generated by $g$ where the DLP is hard.
  Let $q$ have $n$ bits, with $2^{n-1} < q < 2^n$.

  Let $\mathit{iv}$ be a fixed $n$-bit integer.
  The hash function $H_{\mathit{iv}}$ is then defined as follows:
  \begin{enumerate}
    \item Given an input $m$ of length $l\cdot n$ bits with some integer $l$, split it into $n$-bit blocks $m_1, m_2, \dots, m_l$.
    Each block $𝑚_i$ can be viewed as an $n$-bit integer.
    \item Let $h_0 \gets g^{\mathit{iv}}$.
    Let $h_i \gets h^{m_i}_{i-1}$ for all $i\in\{1,\dots,l\}$.

    I.e. $h_1 = \bigl(g^{\mathit{iv}}\bigr)^{m_1}$, $h_2 = \bigl(g^{\mathit{iv} \cdot m_1}\bigr)^{m_2}$, \dots
    \item $H_{\mathit{iv}}(m) \overset{\mathit{def.}}{=} h_l$.
    That is, the hash of $m$ under $H_{\mathit{iv}}$ is $h_l$.
  \end{enumerate}

  Answer the following:

  \begin{enumerate}
    \item Show that this hash function does not have second pre-image resistance for $l \ge 2$.
    \textit{(2 points)}

    Hint: multiplication is commutative.

    \item Explain why this hash function is first pre-image resistant.
    \textit{(2 points)}
  \end{enumerate}
\end{task}

\begin{task}{Authentic$\infty$tion}
  Let $H$ be a publicly known collision resistant hash function with $n$-bit outputs.
  Let $\Enc$ be the encryption algorithm of a secure block cipher with a block and key-size of $n$ bits.

  Consider the keyed MAC construction given by:
  \[
    \MAC_{\sk}(m) \overset{\textit{def.}}{=} \Enc_{H(m)}(\sk).
  \]
  That is, instead of encrypting $H(m)$ with the secret key $\sk$, $H(m)$ is used as the block cipher encryption key, and the secret $\sk$ is encrypted instead.

  \begin{enumerate}
    \item Show that this MAC scheme is not secure against universal forgery attacks.
    That is, find an efficient way to forge a tag on \emph{any} message.
    \textit{(2 points)}

    \item Why use $H(m)$ instead of using $m$ directly as the encryption key?
    \textit{(1 point)}
  \end{enumerate}
\end{task}

\begin{task}{Mind the group (bonus)}
  Let $g = 5$ be the ElGamal group generator in the field $\ZZ_{23}$.
  Show that ElGamal does not have IND-CPA in this group.

  \textit{(2 points)}
\end{task}

\end{document}
